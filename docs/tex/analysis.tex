\chapter{Аналитическая часть}

\section{Методы визуализации волн}

Существует три группы методов моделирования волн:

\begin{itemize}
    \item процедурные;
    \item методы высотного поля;
    \item методы на основе частиц.
\end{itemize}

\subsection{Процедурные методы}

В процедурных методах для представления движения волн используются периодические функции. В ранних работах в качестве такой функции выступала циклоида \cite{orbit-procedure}, далее стали использовать синусоиду \cite{spectrum-darles}. Наложение периодических функций, изменяющихся во времени, создает волновую поверхность. Точка на такой поверхности описывает замкнутую круговую орбиту. Для создания различных волновых эффектов изменяют параметры уравнений орбиты, например, радиус, фазовый угол.

Процедурные методы чаще всего используют при визуализации масштабных волн океана. Преимуществом процедурного моделирования является возможность точно контролировать движение волнового спектра. Недостаток данных методов - сложность получения правильного взаимодействия волн с погруженными телами и границами.  

Выделяют следующие процедурные методы:

\begin{itemize}
    \item метод, основанный на модели Герстнера \cite{orbit-procedure};
    \item спектральные подходы \cite{spectrum-tessendorf}\cite{spectrum-darles}.
\end{itemize}

\subsection{Методы поля высот}

В случаях, когда визуализировать необходимо только поверхность воды,а не весь объем водоема, рассматривают волновое уравнение. Волновая поверхность представляется в виде двумерной функции - поля высот \cite{field}. 

Такое упрощение обладает важным преимуществом - снижением вычислительных затрат, что означает повышение скорости моделирования. Кроме того методы поля высот гибко обрабатывают препятствия. Но при такой модели в каждой точке поверхности известно только одно значение высоты, что означает одинаковую скорость распространения всех волн, так, невозможно создать обрушивающиеся волны.

% TODO: 3D-решатели, с ними детальное улавливание эффектов.

\subsection{Методы на основе частиц}

В следующих методах моделирования волновой поверхности вода представляется как система частиц. Частицы движутся в соответствии с законами механики и обладают физическими величинами, т. е. задана функция. В определенный момент времени при помощи интерполяции можно получить значение этой функции в произвольной точке.

Для создания реалистичного изображения необходимо большое количество частиц, поэтому даннные методы используются при визуализации небольшого количества воды.  

Наиболее распространнёные методы на основе частиц:

\begin{itemize}
    \item гидродинамика сглаженных частиц (SPH) \cite{sph};
    \item полунеявный метод движущихся частив (MPS) \cite{mps}.
\end{itemize}

Комбинация методов полей высот и методов, основанных на частицах, позволяет обходить недостатки отдельных и создавать различные эффекты \cite{shallow}\cite{large-small}.

\section{Модели волны и предмета}

\subsection{Модель волны}

Опишу как будет представляться волна. Сделаю вывод о том, что потребуется для модели волны.

\subsection{Модель предмета}

Опишу как будет представляться предмет. Сделаю вывод о том, что потребуется для его создания.

\section{Анализ алгоритмов удаления невидимых линий и поверхностей}

Сравню алгоритмы удаления невидимых линий и поверхностей по критериям. В выводе подраздела выберу лучший.

\section{Анализ методов закрашивания}

Сравню методы закрашивания по параметрам. В выводе подраздела выберу победителя.

\section{Модель освещения}

Опишу модели освещения. Выберу подходящую.

В конце подытожу все.