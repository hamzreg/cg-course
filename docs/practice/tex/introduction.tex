\chapter*{Введение}
\addcontentsline{toc}{chapter}{Введение}

Одними из областей применения компьютерной графики являются  фильмы и компьютерные игры. В данных отраслях компьютерная графика решает задачи представления объектов и процессов реальной жизни. Способ визуализации предметов и действий оценивают по таким характеристикам, как реалистичность результата и время выполнения. Для повышения указанных параметров создаются новые алгоритмы и методы моделирования.

Представление жидкости - одна из наиболее распространненых моделей, которую реализуют в дизайне компьютерных игр и кинематографических спецэффектах: моделирование водоёмов, процессов смешивания и движения водных потоков \cite{large-water-bodies}. Важным физическим явлением для создания водоемов является образование волн на поверхности воды \cite{ocean-simulation}. Для получения более точного изображения визуализируют круговые волны, наложение волн, их прозрачность.

Поверхность воды рассматривают в системе с окружающим миром: при контакте с предметами и препятствиями. Особую сложность для моделирования представляют волны, образованные при движении объектов по воде \cite{dispersion-kernels}.

Цель работы - разработать программное обеспечение, которое предоставляет возможность визуализации линейных волн, образованных при взаимодействии поверхности воды с движущимся твердым телом.

Для достижения поставленной цели требуется решить следующие задачи:

\begin{itemize}
	\item проанализировать методы и алгоритмы, моделирующие волновую поверхность и предмет на воде; 
	\item выбрать алгоритмы и структуры данных для визуализации описанной выше системы; 
	\item реализовать выбранные алгоритмы моделирования;
	\item провести сравнение физических характеристик разработанной модели и реальных волн, взаимодействующих с объектом.
\end{itemize}
