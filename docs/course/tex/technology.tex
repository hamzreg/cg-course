\chapter{Технологическая часть}

В данном разделе описываются средства разработки программного обеспечения и детали реализации.

\section{Средства реализации}

Разработка программного обеспечения выполнялась при помощи языка программирования Python. Выбор языка программирования обусловлен его преимуществами:

\begin{itemize}
	\item наличие библиотек необходимых для реализации всех требуемых функций;
	\item возможность использования объектно-ориентированного подхода.
\end{itemize}

Для написания вершинных и фрагментных шейдеров графического контейнера OpenGL использовался язык GLSL. Данный язык шейдеров был выбран из-за следующих характеристик:

\begin{itemize}
	\item контроль графического потока без использования машинно-зависимых языков;
	\item высокая производительность;
	\item наличие дополнительных функций и типов данных для работы с матрицами и векторами.
\end{itemize}

Пользовательский интерфейс программного обеспечения создан при помощи Qt Designer, так как инструмент позволяет быстро разработать интерфейс.

Для ускорения процесса создания программного обеспечения в качестве среды разработки выбран текстовый редактор Visual Studio Code.

\section{Реализация алгоритмов}

В листинге \ref{lst:wave} представлен алгоритм образования волн при движении твердого тела, в листинге \ref{lst:sphere} --- алгоритм перемещения предмета.

\begin{center}
\captionsetup{justification=raggedright,singlelinecheck=off}
\begin{lstlisting}[label=lst:wave,caption=Алгоритм образования волн при движении предмета]
in vec2 coord;

const float w = 1.985;

uniform sampler2D currTexture;
uniform sampler2D prevTexture;
uniform bool moveSphere;
uniform float sphereRadius;
uniform vec3 nowCenter;
uniform vec3 oldCenter;
uniform float step;

float volumeInSphere(vec3 sphereCenter)
{
    vec3 toCenter = vec3(coord.x * 2.0 - 1.0, 0.0, coord.y * 2.0 - 1.0) - sphereCenter;
    float t = length(toCenter) / sphereRadius;
    float dy = exp(-pow(t * 1.5, 6.0));
    float ymin = min(0.0, sphereCenter.y - dy);
    float ymax = min(max(0.0, sphereCenter.y + dy), ymin + 2.0 * dy);

    return (ymax - ymin) * 0.1;
}

void main()
{
    vec2 dx = vec2(step, 0.0);
    vec2 dy = vec2(0.0, step);

    float average = (texture(currTexture, coord + dy).y +
                     texture(currTexture, coord - dy).y +
                     texture(currTexture, coord + dx).y +
                     texture(currTexture, coord - dx).y  ) * 0.25;

    float prev = texture(prevTexture, coord).y;
    float h = (1.0 - w) * prev + w * average;

    if (moveSphere)
    {
        h += volumeInSphere(oldCenter);
        h -= volumeInSphere(nowCenter);
    }

    gl_FragColor = vec4(0.0, h, 0.0, 1.0);
}
\end{lstlisting}
\end{center}

\begin{center}
\captionsetup{justification=raggedright,singlelinecheck=off}
\begin{lstlisting}[label=lst:sphere,caption=Алгоритм перемещения предмета]
def moveObject(self):
	if self.moveSphere:
		if self.route == POSITIVE:
			sign = 1
		else:
			sign = -1

		self.change += sign * self.time * self.velocity

	if START_SPHERE_CENTER.x + self.change + self.sphereRadius >= POSITIVE_BORDER:
		self.route = NEGATIVE
	elif START_SPHERE_CENTER.x + self.change - self.sphereRadius <= NEGATIVE_BORDER:
		self.route = POSITIVE
\end{lstlisting}
\end{center}

\section*{Вывод}

Были описаны инструменты разработки программного продукта, и представлены алгоритмы моделирования волнового процесса в результате движения объекта.