\chapter*{Введение}
\addcontentsline{toc}{chapter}{Введение}

Компьютерная графика применяется в киноиндустрии и разработке компьютерных игр. Методами компьютерной графики решаются задачи представления объектов и процессов реальной жизни в виртуальной реальности. Способ визуализации предметов и действий оценивают при помощи характеристик --- реалистичности результата и скорости выполнения. Для повышения указанных параметров создаются новые алгоритмы и методы моделирования.

Представление жидкости --- модель, которую реализуют в дизайне компьютерных игр и кинематографических спецэффектах: моделирование водоёмов, процессов смешивания и движения водных потоков. Важным физическим явлением для создания реалистичного водоема является образование волн на поверхности воды. Для получения более точного изображения визуализируют круговые волны, наложение волн, их прозрачность.

Поверхность воды рассматривают в системе с окружающим миром: при взаимодействии с предметами и препятствиями. Особую сложность для моделирования представляют волны, образованные при движении объектов по воде.

Цель работы --- разработать программное обеспечение, которое предоставляет возможность визуализации волн, образованных при взаимодействии поверхности воды с движущимся твердым телом.

Для достижения поставленной цели требуется решить следующие задачи:

\begin{itemize}
	\item изучить волновой процесс;
	\item формально описать структуру системы, состоящей из поверхности воды и источника волн;
	\item проанализировать методы и алгоритмы, моделирующие волновой процесс;
	\item выбрать алгоритм и структуры данных для визуализации описанной выше системы;
	\item реализовать выбранный алгоритм моделирования;
	\item провести анализ производительности программного обеспечения.
\end{itemize}
